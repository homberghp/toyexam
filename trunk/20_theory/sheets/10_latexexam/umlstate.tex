\section{UML state representation}
\begin{frame}{UML State representation}
  Uses and application
  \begin{itemize}

  \item In UML state behaviour is modeled with state charts (diagrams)
    or activity diagrams
  \item State charts are mostly used in reactive systems (systems that
    wait for and react to asynchronous input changes)
  \item Activity diagrams are used in systems where activies are a
    sequential pattern of synchrous activities
  \item Pattern matchers and compilers are an other large application
    area of state diagrams
  \end{itemize}
\end{frame}
\subsection{State chart syntax}
\begin{frame}{State chart components}
  \begin{itemize}
  \item State: A stable condition that persists for a significant period
    of time (disjoint with other such conditions)
  \item Transition: The change of one state into another in
    practically no time usually initiated by an event. When state
    remains the same it is a ``self transition''
  \end{itemize}
\end{frame}

\begin{frame}{Statechart components (Cont’d)}
  \begin{itemize}

  \item Event: A time discrete occurrence of interest
  \item Guard: Conditions that must be met for a transition to be taken
  \item Action: Quick operation executed in practically no time establishing a relevant change or result
  \end{itemize}
\end{frame}

\begin{frame}{Execution of actions}
  Actions can be executed when:
  \begin{itemize}
  \item A state is entered. Specified with \Okis{\texttt{entry:}} in the state container
  \item A transition is made. Specified besides the transition
  \item A state is left. Specified with \Okis{\texttt{exit:}} in the state container.
  \end{itemize}
\end{frame}

\subsection{Transition notation}
\begin{frame}{Specification of transitions}
  \textbf{  Labelsyntax: eventname [guard] / actionlist }
  \begin{description}
  \item[eventname] Name of the event triggering the transition
  \item[guard] Boolean expression that must evaluate to \Okis{TRUE} for the transition to be taken
  \item[actionlist] Dot-comma separated list of actions executed as
    a result of the transtion taken
  \end{description}
\end{frame}

\begin{frame}{Initial and terminal pseudo states}
  \begin{itemize}
  \item Are almost, but not quite, states. States with a transient
    character (i.e. not triggered by event).
  \item We know the following pseudo states:\\
    \Okis{Initial state}: Indicates the initial or default state.
    The associated transition to must be made unconditionally (no
    guard). Drawn as filled dot. \includegraphics[height=.8\baselineskip]{../figures/stateInitial}\\
    \Okis{Terminal state}: Indicates destruction of the object
    (at the outermost level). Drawn as encircled dot. \includegraphics[height=.8\baselineskip]{../figures/stateFinal}\\
    \Okis{Choice state}: Selects the next state dependent on some
    condition. Used to simplify the diagram by reducing the number
    of outgoing transitions of a state. \includegraphics[height=.8\baselineskip]{../figures/stateChoice}
  \end{itemize}
\end{frame}

% \begin{frame}{/ FanOff;CoolOff;HeatOff evOff

%     STANDBY evTimer / FanOff Delay

%     evFan / CoolOff;FanOff

%     FanOff / StartTimer(30 sec);HeatOff

%     evFan / FanOff evFan / FanOn FanOn / StopTimer;HeatOn

%     COOLING HEATING evTemp[ switch == cool ] / CoolOn

%     evTemp[ switch == heat ] / heatOn

%     evTemp / CoolOff FAN evTemp / heatOff

%     12-10-2004

%     (c) Fontys, Cees van Tilborg

%     32
\subsection{Composite states}
\begin{frame}{Composed states}
  \begin{itemize}
  \item A Super state or container state may be composed of two or more
    (sequential) sub states
  \item If state machine reaches super state S, consisting of two
    (sequential) substates S11 and S12, then it reaches either sub state
    S11 \textbf{OR} sub state S12.
  \item If a state is not further composed of other states it is called
    a leaf state.
  \item Group transition: A transition connected to the edge of a
    composed state (or: non leave state).
    \begin{itemize}
    \item Incoming: The sub state entered is determined by initial (or:
      history if present) pseudo state
    \item Leaving: The last actual sub state
      is stored in the history connector (if present at all)
    \end{itemize}
  \end{itemize}
\end{frame}

\begin{frame}{flat std example}
  \begin{center}
    \includegraphics[height=65mm]{../figures/CruiseControlSTD1}
  \end{center}
\end{frame} % TIL 34 

\begin{frame}{Hierarchical std example}
  \begin{center}
    \includegraphics[height=65mm]{../figures/CruiseControlSTD2}
  \end{center}
\end{frame}% TIL       35

\begin{frame}{History pseudo state}
  \begin{itemize}
  \item First time, if no history exists, the connected state is
    initially entered (like the initial pseudo state).
  \item Later on the last occupied sub state is entered. Two
    variants:
    \begin{itemize}
    \item Shallow history: Applied to the newly entered state only
    \item Deep history (with star): Applied to the newly entered state
      and all its child states
    \end{itemize}
  \end{itemize}
\end{frame}

\begin{frame}{State machine operation}
  \begin{itemize}
  \item In a state machine the state changes only due to an event
    occurrence
  \item If an event occurs in a state where an leaving transition
    exists with that particular event name next to it, the transition
    is triggered. A triggered transition is taken only:
    \begin{itemize}
    \item when no guard is present
    \item if a guard is present and the guard conditions are met.
    \end{itemize}
  \item When the transition is taken a chain of actions is executed:
    The actions from action lists associated with the states left
    (inside out), the transition and the states entered (outside in).
  \item Also when the transition is taken the state is changed unless it
    is a self transition. In that case the state remains the same, but
    some action can be taken.
  \end{itemize}
\end{frame}

\begin{frame}{Order of actions}
  \begin{center}
    \includegraphics[height=65mm]{../figures/OrderOfActions}
  \end{center}
\end{frame}
 % TIL 38

\begin{frame}[shrink]{Order of actions with direct to inner state}
  With a direct trasition to an inner state, the initial state is bypassed.
  \begin{center}
    \includegraphics[height=60mm]{../figures/OrderOfActions2}
  \end{center}
\end{frame}

\begin{frame}{Order of actions with direct to inner state(2)}
  This also applies to diagrams with history states.
  \begin{center}
    \includegraphics[height=55mm]{../figures/OrderOfActions24}
  \end{center}
  Quiz: Criticize this diagram.
\end{frame} % TIL 40


\begin{frame}{Recommendations}
  \begin{itemize}
  \item {\color{red}Wrong}
    \begin{itemize}
    \item Unlabeled transitions (no event)
    \item Guard on initial/history action list
    \item State with
      \begin{itemize}
      \item only incoming transitions
      \item only transitions leaving
      \item more transitions leaving responding to the same event with
        no or non mutual exclusive guards
      \item Overlapping guard conditions
      \end{itemize}
    \end{itemize}
  \item {\color{orange}Better not}
    \begin{itemize}
    \item Apply an entry action list and als an initial action list
    \item Add an action list to a history state
    \item Use the deep history state (it is not clear on lower levels)
    \end{itemize}
  \end{itemize}
\end{frame}

\begin{frame}{More UML Modeling}
  Besides statecharts we will use:
  \begin{itemize}
  \item Use Case Diagram/descriptions
  \item Class Diagram
  \item Sequence Diagram
  \end{itemize}
  \textit{
  A consistent and coherent set of diagrams and descriptions together
  form a model of your system. This model helps you to make sure you
  understand the problems to be solved and the needs to be fulfilled and
  to make the step towards the implementation smaller.}
\end{frame}

\begin{frame}{Use case diagram/description}
  \begin{itemize}

  \item Use case
    \begin{itemize}
    \item Depicted by an oval. Each use case has an associated
      description. Described is the interaction of an actor with the
      system to obtain some result from the system or to establish a
      change to the system without revealing the systems internal
      structure.
    \end{itemize}
  \item  Actor: Participant from outside the system
  \item The \Stereotype{uses} Relation (default) \Stereotype{extend}
    Relation \Stereotype{includes} Relation
  \end{itemize}
\end{frame}


% \begin{frame}{Withdraw Funds \Stereotype{includes}}

% \Stereotype{includes} Query Account Customer \Stereotype{includes} Validate PIN

% Add Cash Transfer Funds

% Startup

% Operator

% Shutdown

% 12-10-2004

% (c) Fontys, Cees van Tilborg

% 45

\begin{frame}{Use case description}
  A use case is described in a table, showing:
  \begin{itemize}
  \item Name (as in diagram!)
  \item Summary
  \item Dependency (extend/include relations)
  \item Actors
  \item Preconditions (assumptions)
  \item Description
  \item Alternatives (another or exceptional sequence)
  \item Post-conditions (Result)
  \end{itemize}
\end{frame}

\begin{frame}{Class diagram}
  \begin{itemize}
  \item Classes
  \item Attributes
  \item Operations (or: methods)
  \item Relations
    \begin{itemize}
    \item Associations (bidirectional, unidirectional)
    \item Composition and aggregation (whole-part)
    \item Inheritance (Generalization/specialization)
    \item Realization (Inheritance of abstract classes) Dependency
    \end{itemize}
  \item Multiplicity (cardinality)
  \end{itemize}
\end{frame}


\begin{frame}{Aggregation and composition}
  \begin{itemize}
  \item Whole part relationship
  \item The \textbf{aggregation} relationship expresses a collection
    (e.g. books in library) where as the \textbf{composition} expresses a more
    rigid relation (e.g. a cow has four legs)
  \item Major difference is the lifetime coupling
    \begin{description}
    \item[Aggregation] Parts may be constructed and/or destructed during the lifetime of the whole object
    \item[Composition] Parts are constructed as part of the
      construction of the whole and destructed as part of destruction
      of the whole object
    \end{description}
  \end{itemize}
\end{frame}

\begin{frame}{Aggregation and composition (cont’d)}
  \begin{itemize}

  \item Composition implementation:
    \begin{itemize}
    \item Part is a member variable of whole
    \item Part is instantiated in constructor of whole and discarded
      in the destructor of the whole object.
    \end{itemize}
  \item Aggregation implementation:
    \begin{itemize}
    \item Part may be instantiated or discarded from any method or
      operation of the whole object as required. However all existing
      parts are discarded from the destructor of the whole object
    \end{itemize}
  \end{itemize}
\end{frame}


% \begin{frame}{Specification: Properties of the relation

% 12-10-2004

% (c) Fontys, Cees van Tilborg

% 51

% \begin{frame}{Composition: Aggregation plus containment by value

% 12-10-2004

% (c) Fontys, Cees van Tilborg

% 52

\begin{frame}{Aggregation and composition (Cont’d)}
  \begin{itemize}
  \item The cardinality of a composition relation should be fixed since
    the lifetime is tightly coupled to the whole lifetime. If
    cardinality is omitted, it defaults to one.
  \item The cardinality of an aggregation should not be fixed. The
    number of parts may change during the lifetime of the whole
    object. If cardinality is omitted it defaults to $0\ldots*$
  \end{itemize}
\end{frame}

\begin{frame}{Sequence/collaboration diagram}
  \begin{itemize}
  \item Shows dynamic interaction between objects (class instances),
    ordered in a time sequence
  \item The sequence is usually initiated by an actor and eventually
    brings a result back to the actor
  \item Only the cluster of objects involved is shown and together have
    to establish a specific task
  \item Several sequences may be required to show different scenarios
    (e.g. in case of an error) for a specific task
  \item The collaboration diagram is equivalent to a sequence diagram,
    it shows another representation of the same interaction
  \end{itemize}
\end{frame}

\begin{frame}{Diagram relations (some examples)}
  \begin{itemize}

  \item An incoming arrow on an class instance in the sequence diagram implies that:
    \begin{itemize}
    \item That class exists on the class diagram
    \item That class on the class diagram shows an operation with the
      same name and signature
    \item That class is associated with the class instance on the
      other end of the arrow
    \end{itemize}
  \item A statechart specifies the state behaviour of a class,
    therefore it must always be clear to what class a statechart
    belongs
  \end{itemize}
\end{frame}

\begin{frame}{Implementing model diagrams}
  \begin{itemize}
  \item If syntactical and semantic rules are clear, diagrams can
    be implemented
  \item If rules are more formal and strict, the implementation can
    be generated in an automated way
  \item Upper case tools use this to generate code from the model
    (for a certain execution environment)
  \end{itemize}
\end{frame}

\begin{frame}{Example std}
\includegraphics[height=70mm]{../figures/StateMachineDiagram1}
\end{frame}
% \begin{frame}{CSample + AttribA : Boolean = false # AttribB : Integer = 0 - AttribC : Integer = 0 + MethodA(par1 : Boolean = false) : Boolean # MethodB(p1 : Double = 0) : Boolean - MethodC()

% More and less detailed views are available in Rose.

% C++ style class definition: class CSample { public: CSample(); ~CSample(); bool MethodA(bool par1 = false); bool AttribA; protected: int AttribB; bool MethodB(double p1 = 0.0); private: int AttribC; void MethodC(); }

% Class
% 58

% 12-10-2004


% \begin{frame}{Aggregation}

% The contained classes (parts) are not shown in the attributes field of the containing class (whole).

% Composition

% CW hole1

% CW hole2

% 0..200 CPart1A CPart2A

% 1

% 2 CPart2B

% C++ Style: class CWhole1 { public: CWhole() { npart1a = 0; for (int i=0;i<200;i++) part1a[i]=NULL; } ~CWhole() { while (--npart1a >= 0) { if (part1a[npart1a] != NULL) delete part1a[npart1a]; } } bool addpart(int& id) { //when there is a free spot, //create part and add it } bool rempart(int id) { //remove identified part, when not null } private: npart1a; CPart1A *part1a[200]; };

% C++ Style: class CWhole2 { private: CPart2A part2a; CPart2B part2b[2]; }

% Alternative is to create objects in constructor of Whole2 and to delete them in the ddestructor of W hole2

% Aggregation/Composition
% 59

% 12-10-2004

% (c) Fontys, Cees van Tilborg

% \begin{frame}{bidirectional association.

% unidirectional association.

% C3

% C4

% C1

% C2

% C++ Style: class C3 { private: C4 *p; }

% C++ Style: class C4 { private: C3 *p; }

% C++ Style: class C1 { private: C2 *p; }

% public role

% C5

% +theC6

% C6

% C++ Style: class C5 { public: C6 *theC6; }

% Association
% (c) Fontys, Cees van Tilborg 60

% 12-10-2004

% \begin{frame}{Company

% +Empoyer +Employee

% Person

% Job - salary - Ident + getjobdata() + setjobdata()

% Association class
% \end{frame} 61

% \begin{frame}{CArticle In client.h and CSupplier.h a forward declaration is inserted: class CArticle;

% CSupplier CClient +buyer + buy() +seller + takeorder() + handout() + acceptmoney()

% In client.c an include statement is inserted: #include "CClient.h" #include "CSupplier.h" CClient::buy(CArticle p) { CSupplier &seller=p.getsupplier(); seller.takeorder(p); }

% Dependency
% \end{frame} 62

% \begin{frame}{Genralization/specialization. Is a kind of . . .

% General CBase

% CDerived1

% CDerived2

% Special C++ Style notation: class CBase { } class CDerive1 : CBase { } class CDerive2 : CBase { }

% Inheritance
% 63

% 12-10-2004

% (c) Fontys, Cees van Tilborg

% \begin{frame}{Cannot be instantiated CAbstract + hi()

% CConcrete Can be instantiated

% C++ Style: class CAbstract { public: virtual void hi() =0; } class CConcrete : public CAbstact { public: void hi() { cout \Stereotype{ "hello world" \Stereotype{ endl; } }

% Realizes
% 64

% 12-10-2004

% (c) Fontys, Cees van Tilborg

% \begin{frame}{/ a0 S1 entry/ a1 e1 / a6 e1 e3 / a5 S2 exit/ a2 e6 / a13

% Implementing a statechart
% e5 / a7

% S3 entry/ a3 exit/ a4 S31 entry/ a9 exit/ a10 e2

% e4

% e6 / a8

% e2 e2[ g==true ] / a12 S32

% S33

% e2[ g==false ] / a11

% H

% 12-10-2004

% (c) Fontys, Cees van Tilborg

% 65

\begin{frame}[shrink]{States in table like notation}
\texttt{
Initial: S3H=S32 \\
T1 SI:{a0;a1}:S1\\
T2 S1:e1:{a6}:S2 \\
T3 S1:e5{a7;a3;[S3H==S31]a9}:S3H \\
T4 S2:e3:{a2;a5;a3;a9}:S31 \\
T5 S2:e1:{a2;a1}:s1\\
T6 S2:e6:{a2;a13;a3;[S3H==S31]a9}:S3H \\
T7 S2:e4:{a2}:ST \\
T8 S31:e6:{a10;a4;a8}:S2 \\
T9 S31:e2:{a10}:S32 \\
T10 S32:e6:{a4;a8}:S2 \\
T11 S32:e2[g==true]:{a12;a9}:S31 \\
T12 S32:e2[g==false]:{a11}:S33 \\
T13 S33:e6{a4;a8}:S2 \\
T14 S33:e2{a9}:S31}
\end{frame}
\lstset{basicstyle={\scriptsize\ttfamily}}
\begin{frame}{Simple state implementation}

  Nested state, then event. Const definitions
  \lstinputlisting[firstline=1,lastline=15]{../code/java/simplestate/SimpleStateMachine.java}
\end{frame}  

\begin{frame}{Simple state implementation, cont'd}
  simple state
  \lstinputlisting[firstline=17,lastline=35]{../code/java/simplestate/SimpleStateMachine.java}
\end{frame}  

\begin{frame}{Simple state implementation, cont'd}
  nested state
  \lstinputlisting[firstline=36,lastline=48]{../code/java/simplestate/SimpleStateMachine.java}
\end{frame}  

\begin{frame}{State impl example 2}
  Having a method per event reduces the nesting level of the state
  machine.
  
  This is the same as event then evaluate state.
  \lstinputlisting[firstline=12,lastline=27]{../code/java/simplestate/SimpleEventMethods.java}
  
  
\end{frame}
\begin{frame}{State impl example 2}
  \lstinputlisting[firstline=28,lastline=48]{../code/java/simplestate/SimpleEventMethods.java}
\end{frame}
\begin{frame}{Implementing statecharts}
  \begin{itemize}
  \item Presented are two simple examples of statechart implementation
    strategies, you will see them back during the laboratory work
  \item True object oriented implementations (e.g. based on the state
    pattern) will be introduced during the patterns part.
  \end{itemize}
\end{frame}
