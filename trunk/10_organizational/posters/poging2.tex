\documentclass[22pt,a1paper, portrait,
blockverticalspace=10mm, colspace=12mm]{tikzposter} %Options for format
                                %can be included here
\usepackage[utf8]{inputenc}
\usepackage{helvet}
\usepackage{hyperref}
\usepackage{breakurl}
 % Title, Author, Institute
\title{A M C : Less work, better exams}
\newcommand\alert[1]{{\color{red}#1}}
\newcommand\okis[1]{{\color{green!40!black}#1}}
\author{Pieter van den Hombergh}
\institute{Fontys Hogeschool voor Techniek en Logistiek, Team Informatica}
%\titlegraphic{LogoGraphic Inserted Here}

 %Choose Layout
\usecolorpalette{GreenGrayViolet}

%\usetheme{Wave}
\usetheme{Envelope}
\begin{document}
\sffamily

 % Title block with title, author, logo, etc.
\maketitle[width=.95\textwidth]
 % \block{Title}{Auto multiple choice}
\begin{columns}


\column{0.49}% Width set relative to text width
\block{Robust exam setting}{
\begin{itemize}
\item There is nothing more robust then a paper and pen/pencil \includegraphics[height=\baselineskip]{pencil-500.png} exam,
  exam-wise.
\item Both paper and pencil always work or are easily replaced.
\end{itemize}
}
\block{Prevent Cheating}{
  Give each candidate a personalised exam:
  \begin{itemize}
  \item Name and ID pre-printed.
    \begin{itemize}
    \item Prevents exam question leaking.
    \end{itemize}

  \item Randomise questions per student.
  \item Randomise answer choice per question.
  \item Parametric questions and choices.
  \end{itemize}
}\note[radius=105mm,angle=-10,rotate=8]{\raggedright{}A fair exam setting is in the interest of students}

\block{Quick results\ldots}{
even with \okis{large exam volume}.
  \begin{itemize}
  \item Scan the answer sheets.
  \item  Automatic correction to grades.
  \item Automation prevents boredom \includegraphics[height=20pt]{Smiley.png}
  \item Prefers, but not limited to, multiple choice.
    \begin{itemize}
    \item Improves objectiveness.
    \item Multiple question types are possible.
    \end{itemize}
  \end{itemize}
}
\note[radius=100mm,angle=18,rotate=-2]{\raggedright{}Corrected and in
  \textbf{{\color{red}progress}\ } before student is home}

\block{Less work, Better exams}{
  \begin{itemize}
  \item when properly used, the exam will be less work
  \item and be of a better quality, even optically.
  \end{itemize}
}
\block{Exam (Quality) Analysis}{
  \begin{itemize}
  \item Analyse the exam and question quality
    \begin{itemize}
    \item is the question fair?
    \item does a question properly discriminate?
    \end{itemize}
  \item \okis{Detailed spreadsheet} output for scores
    \begin{itemize}
    \item statistics per individual question (even over multiple exams)
    \item simple copy-submitting results to \textbf{{\color{red}progress}}
    \end{itemize}
  \end{itemize}
}

\column{.49}
\block{Reusable questions}{
  \begin{itemize}
  \item Over time you will gather a catalogue of exams in a ``database''.
  \item Easy to create and maintain variants of questions.
  \item No copy and \textbf{W}aste problem.
  \item Easy sharing work with colleagues.
  \item The used questions are in a simple list.
  \end{itemize}
}

\block{Costs}{
  \begin{itemize}
  \item \textbf{Auto Multiple Choice} (AMC) is open source.
  \item The main author is Alexis Bienvenüe.
  \item see \url{http://home.gna.org/auto-qcm/}
    \begin{itemize}
    \item The developers are quite responsive.
    \end{itemize}

  \item It uses \LaTeX\footnote{there is a simpler, less powerful
      mode} so you need some smartness \includegraphics[height=20pt]{Smiley.png}.
  \end{itemize}
}
\note[radius=112mm,angle=40,rotate=-5]{\raggedright{}\textbf{AMC} 
  is {\Large{}\okis{\textbf{free}}}, just add some thinking\ldots}

\block{Fontys Informatica Venlo}{
  \begin{itemize}
  \item The informatica team of FHTenL has gained experience with
    several years of AMC exams.
  \item \LaTeX is well known (t)here.
\end{itemize}
}
\block{Runs on\ldots}{
  \begin{itemize}
  \item Linux, tested and current practice.
    \begin{itemize}
    \item Tested at the informatics team Venlo.
    \item Runs on a server, to share maintenance work, experience and best practices.
    \item Can be run in a virtual machine,\\e.g. for running under Windows.
    \end{itemize}
  \item Apple Mac-OS-X.
    \begin{itemize}
    \item Currently \alert{un}tested in Venlo.
    \end{itemize}
  \end{itemize}
}\note[radius=135mm,angle=32,rotate=25]{\raggedright{}Write your questions on any platform}

\block{Campus Venlo\\ Innovation in\\ Education and Examination}{
\centering  {\textbf{\color{green!20!black}Towards less effort with better results}!!}
}
\end{columns}
% \setlength{\unitlength}{1mm}
%  \begin{picture}(300,330)(0,380)
%    \put(80,0){\includegraphics[width=100mm]{PencilPaper-752x500.jpg}}
%  \end{picture}
\end{document}
