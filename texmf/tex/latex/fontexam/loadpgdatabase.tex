\providecommand\dbname{webshop}
\providecommand\username{exam}
\providecommand\schemafile{schema.sql}
\subsection{Create and load the postgresql database \dbname}
\label{setupdb}
You must create a postgresql database named \Code{\dbname} and load it
with the initial data. To do that follow these steps:
\begin{enumerate*}\itemsep1pt\parskip0pt\parsep0pt
\item Open the database GUI pgadmin III.
\item Connect to the \Code{exam} database. The user/role name is
  \textbf{\username} with password \textbf{\username}.
\item Open the \textit{servers} node, then open the server node \Code{exam}.
\item In the object browser (left panel with tree like structure)
  right-click on databases node in the tree and select \textbf{New database}
\item In the \textit{new Database} dialog enter \Code{\dbname} in
  the name field and leave the rest to their defaults. Click Ok.
  \begin{itemize}
  \item Click on the \Code{\dbname} database to activate the sql buttons.
  \end{itemize}
\item Select the execute sql button, the one with the magnifying
  glass.
\item Open the file browser (the folder icon).
\item In the folder \Code{\ldots/dbscripts/} you find the file
  \Code{\schemafile}. Open, then execute it using the \textbf{ExecuteScript} button.
\item In the object browser, you can now verify that the script
  executed  successfully, by checking that the \Code{\dbname} database exists
  and the tables, views and sequences as are defined in the
  script. 
\end{enumerate*}

When you want to restore the database to it's original state with
initial table data, you can re-execute the \Code{\schemafile} script in
\textbf{pgAdmin III}.

% To make the jdbc driver available to the projects, you should
% first create a Netbeans Library, named as the one used in the project.
% See the assessment website for the exact details.
