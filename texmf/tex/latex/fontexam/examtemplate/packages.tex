%% packages typical for exams that need to show source code listings.
%% It is good practice to not copy the program code into the latex
%% files, but instead to use \lstinputlisting commands to include the
%% source code. This avoids syntax clashes between latex and source code.
\usepackage{listings}
\usepackage{times}
\usepackage{textcomp}
\usepackage{xcolor}  
\usepackage{caption}
\usepackage{tikz}
\definecolor{lbcolor}{rgb}{0.9,0.9,0.9}
\definecolor{darkblue}{rgb}{0,0,.2}
\definecolor{darkgreen}{rgb}{0,0.2,0}
%% these listing settings define a nice box with a caption around each
%% listing.

\lstset{
  aboveskip={1.5\baselineskip},
  basicstyle=\footnotesize\ttfamily,
  breaklines=true,
  columns=fixed,
  commentstyle=\color[rgb]{0,0.2,0},
  extendedchars=true,
  frame=t,
  framexbottommargin=4pt,
  framexleftmargin=17pt,
  framexrightmargin=5pt,
  identifierstyle=\ttfamily,
  keywordstyle=\color{darkblue}\textbf,
  language=java,
  numbersep=5pt,
  % numberstyle=\tiny,
  prebreak=\raisebox{0ex}[0ex][0ex]{\ensuremath{\hookleftarrow}},
  showspaces=false,
  showstringspaces=false,
  showstringspaces=false,
  showtabs=false,
  stringstyle={\color{darkgreen}\ttfamily\textbf},
  tabsize=4,
  upquote=true,
}
\DeclareCaptionFont{black}{\color{black}}
\DeclareCaptionFormat{listing}{\colorbox[gray]{.9}{\parbox{\textwidth}{\hspace{15pt}#1#2#3}}}
\captionsetup[lstlisting]{format=listing,%
  labelfont=black,%
  textfont=black,%
  singlelinecheck=false,%
  margin=0pt,%
  font={bf,footnotesize}}

%% end of file
