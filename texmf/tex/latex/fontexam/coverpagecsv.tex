% 
% Voorblad voor Fontys HV tentamens
\usepackage{pifont}
\usepackage{multicol}
% \newcommand{\progresscode}{MCM1TT}
\newcommand\tickno{\ding{111}}%{$\circ$}
\newcommand\tickyes{\ding{52}}%{$\bullet$}
\newlength\frontpageboxwidth
\setlength\frontpageboxwidth{172mm}
\providecommand\Papierja\tickno      \providecommand\Papiernee\tickyes
\providecommand\Papierklad\tickno    \providecommand\Papierlijn\tickno
\providecommand\Papierruit\tickno     
\providecommand\Rekenappja\tickno     \providecommand\Rekenappnee\tickyes
\providecommand\Diktaatja\tickno      \providecommand\Diktaatnee\tickyes
\providecommand\Boekja\tickno         \providecommand\Boeknee\tickyes
\providecommand\Anderlmja\tickno     \providecommand\Anderlmnee\tickyes
\providecommand\Inleverenja\tickyes   \providecommand\Inleverennee\tickno
\providecommand\Boeklijst{}
\providecommand\Diktaatlijst{}
\providecommand\Validator{\Lecturer}
\providecommand\ValidatorSignature{signatures/scn_sig.png}
\providecommand\ValidatorPicture{}
% \providecommand\ValidatorPicture{\begin{picture}(0,0)(0,0)\put(0,0){\includegraphics[height=1.5cm]{\ValidatorSignature}}\end{picture}}
\newcommand\goodmark{\raisebox{-.5ex}{\includegraphics[height=2ex]{goodmark.png}}}
\newcommand\badmark{\raisebox{-.5ex}{\includegraphics[height=2ex]{badmark.png}}}
\def\Ondertitel{Software Engineering/\NL{Bedrijfskundige
Informatica}\DE{Wirtschaftsinformatik}\EN{Business Informatics}}
\newcommand\aleermiddelbox{}
\newcommand\Opmerkingen{}
% \newcommand\Doelgroep{definieer Doelgroep!}
\InputIfFileExists{\ExamInstanceDir/date.tex}{}{}
\providecommand\Validator{\Lecturer}
\setlength\unitlength{1mm}
\providecommand\coverpage{%
  \setlength\parindent{10mm}
  \begin{picture}(195,225)(10,20)
    %% Logo
    \put(0,230){\includegraphics[width=1.5in]{logo}}
    \put(22,240){
      \begin{minipage}{.85\frontpageboxwidth}
        \flushright
        % \begin{spacing}{1.5}
        \textbf{\Large\textsf{Hogeschool voor Techniek en
            Logistiek\\ \vspace{.3\baselineskip}
            \large\Ondertitel\\ \vspace{.3\baselineskip}
            \large\examtype}}
        % \end{spacing}
      \end{minipage}
    }
    
    % module identification

    \put(-10,220){
      \textsf{
        \fbox{\framebox[\frontpageboxwidth][c]{
            \NL{
              \begin{tabular*}{.85\frontpageboxwidth}[t]{l@{\extracolsep{\fill}}r}
                Vak: \examtitle&\\
                ProgRESS-code: \examcode & Datum: \examdate\\
                Docent: \Lecturer & Tijd: \examtime\\
                Gecontroleerd door: \Validator \ValidatorPicture& \\
                Opleiding: Informatica & Aantal bladen: \AMCpageref{lastpage}\\
              \end{tabular*}
            }
            \DE{
              \begin{tabular*}{0.85\frontpageboxwidth}[t]{l@{\extracolsep{\fill}}r}
                Fach: \examtitle&\\
                ProgRESS-code: \examcode & Datum: \examdate\\
                Dozent: \Lecturer & Zeit: \examtime\\
                Validiert von: \Validator \ValidatorPicture&\\
                Studiengang: Informatik & Seitenzahl: \AMCpageref{lastpage}\\
              \end{tabular*}
            }
            \EN{
              \begin{tabular*}{.85\frontpageboxwidth}[t]{l@{\extracolsep{\fill}}r}
                Module: \examtitle&\\
                ProgRESS-code: \examcode & Date: \examdate\\
                Teacher: \Lecturer & Time: \examtime\\
                Validated by: \Validator \ValidatorPicture &\\
                Course: Informatica & Page count: \AMCpageref{lastpage}\\
              \end{tabular*}
            }
          }
        }
      }
    }% end put        


    \put(-10,185){
      \fbox{
        \framebox[\frontpageboxwidth][c]{
          \NL{
            \begin{tabular*}{.85\frontpageboxwidth}[t]{llll}
              \textbf{Gegevens m.b.t. leermiddelen} &&&\\
              Papier &\Papiernee Nee& \Papierja Ja en wel &
                                                            \Papierklad gemarkeerd Kladpapier\\
              &&                                   & \Papierlijn Lijntjespapier\\
              &&                                   & \Papierruit Ruitjespapier \\
              Rekenapparaat & \Rekenappnee Nee & \Rekenappja Ja &\\
              Diktaten &\Diktaatnee Nee& \Diktaatja Ja en wel& \Diktaatlijst\\
              Boeken & \Boeknee Nee & \Boekja Ja en wel& \Boeklijst\\
              Andere leermiddelen &\Anderlmnee Nee & \Anderlmja Ja en wel& \aleermiddelbox\\
              Tentamenopgaven inleveren & \Inleverennee Nee & \Inleverenja Ja&\\
            \end{tabular*}
          }
          \EN{
            \begin{tabular*}{0.85\frontpageboxwidth}[t]{llll}
              \textbf{Information w.r.t. materials} &&&\\
              Paper &\Papiernee No & \Papierja Yes and & \Papierklad
                                                         marked scratch paper\\
              &&                                       & \Papierlijn Lined paper \\
              &&                                       & \Papierruit squared paper \\
              Calculator & \Rekenappnee No & \Rekenappja Yes &\\
              Sylabi &\Diktaatnee No& \Diktaatja Yes and& \Diktaatlijst\\
              Books & \Boeknee No & \Boekja Yes and & \Boeklijst\\
              Other study materials &\Anderlmnee No & \Anderlmja Yes and & \aleermiddelbox\\
              Hand in Exam questions & \Inleverennee No & \Inleverenja Yes &\\
            \end{tabular*}
          }
          \DE{
            \begin{tabular*}{0.85\frontpageboxwidth}[t]{llll}
              \textbf{Erlaubte Lehrmittel} &&&\\
              Papier &\Papiernee Nein& \Papierja Ja und zwar &
                                                               \Papierklad markiertes Schmierpapier\\
              &&                                   & \Papierlijn liniertes Papier\\
              &&                                   & \Papierruit kariertes Papier \\
              Taschenrechner & \Rekenappnee Nein & \Rekenappja Ja &\\
              Skripte &\Diktaatnee Nein& \Diktaatja Ja und zwar& \Diktaatlijst\\
              Bücher & \Boeknee Nein & \Boekja Ja und zwar& \Boeklijst\\
              Sonstige Lehrmittel &\Anderlmnee Nein & \Anderlmja Ja und zwar & \aleermiddelbox\\
              Klausurbogen einlieferen & \Inleverennee Nein & \Inleverenja Ja&\\
            \end{tabular*}
          }
        }
      }
    }
    
    \put(-10,101){
      \fbox{\framebox[\frontpageboxwidth][c]{\sffamily
          \begin{minipage}{.85\frontpageboxwidth}%
            \raggedright\parindent\medskipamount
            \begin{center}
              \textbf{\EN{Remarks}\NL{Opmerkingen}\DE{Bemerkungen}}
            \end{center}
            
            \begin{itemize}%
            % \item \EN{Remember to check your name and student number on
            %     the score form.}
            %   \NL{Denk erom naam en studentnummer op het voorblad en
            %     scoreblad in te vullen.}
            %   \DE{Denken Sie daran den Namen und die Studentennummer
            %     auf dem Vorblatt und Antwortblatt aus zu füllen.}
            \item \EN{This exam has \amccount\ multiple choice questions earning a
                maximum of \amcpoints\ points and
                \amcopencount\ open questions earning a maximum of
                \amcopenpoints\ points. The points are summed and
                normalised to a range of 1 to 10.}
              \NL{Dit tentamen omvat \amccount\ multiple choice
                vragen waarmee je \amcpoints\ punten kunt verdienen en
                \amcopencount\ open vragen waarmee je \amcopenpoints\ 
                punten kunt verdienen. De punten worden opgeteld en
                genormaliseerd naar een bereik van 1 to 10. }
              \DE{Diese Klausur enthält \amccount\ Multiple-Choice Aufgaben, mit denen
                man \amcpoints\ Punkte verdienen kann, und
                \amcopencount\ 
                offene Fragen, mit denen \amcopenpoints\ Punkte verdient
                werden können. Die Punkte werden addiert und auf
                einen Bereich von 1 bis 10 normalisiert.}  
            \item \EN{Completely fill out the black boxes with a black
                pencil or felt tip. Thus
                like this:  \goodmark\ 
                and \textit{not} like: \badmark.
                When writing the answers to the open questions to the
                answer sheet, please stay clear of the other boxes.
              }
              \NL{Vul bij het invullen van het
                antwoordblad de hokjes volledig met zacht zwart potlood
                of zwarte viltstift. Dus
                zo: \goodmark\ 
                en niet
                zo: \badmark.                
                Schrijf bij het invullen van de open vragen op het
                antwoordblad niet over de hokjes van de andere antwoorden. 
              }
              \DE{Beim Ausfüllen des Antwortblatts die
                Kästchen vollständig mit Bleistift oder
                Filzschreiber ausfüllen. Also
                so:  \goodmark\ 
                und nicht so: \badmark.

                Schreiben Sie beim Eintragen der offenen Fragen auf dem
                Antwortblatt nicht über die Kästchen der anderen
                Antworten.
              }
            \item{\large \NL{Bij sommige vragen zijn 1 of meer
                  antwoorden mogelijk. Deze opgaven zijn gemerkt met
                  een \multiSymbole{}. Waar het \multiSymbole\ 
                  ontbreekt is maar één antwoord correct.}
                \EN{With some questions one or more answers apply.
                  Such questions as marked wit a
                  \multiSymbole{}. Where the \multiSymbole\ is missing,
                only one answer is correct.}
                \DE{Bei manchen Fragen  sind eine oder mehrere
                  Antworten möglich. Solche Fragen sind mit einem
                  \multiSymbole{} gekennzeichnet. Wo das
                  \multiSymbole\ fehlt ist nur eine Antwort richtig.}}
            \item \EN{\textbf{Hint:} The multiple choice exam has a
                separate answer form. Use the exam pages to write down your tentative
                answers. Once you are satisfied with your the answers,
                copy them to the answer form. Please do not fold, crush or smear the
                correctors copy.}
              \NL{\textbf{Tip:} Het examen heeft een los antwoordblad.
                Schrijf je voorlopige antwoorden op het examenpapier
                of kladpapier. Zodra je tevreden bent met je
                antwoorden kopieer je deze naar het losse
                antwoordblad. Zorg er voor dat het antwoordblad niet gekreukt of
                vuil raakt.}
              \DE{\textbf{Hinweis:} Diese Klausur enhält ein separates
                Antwortblatt. Schreiben Sie Ihre vorläufigen Antworten zunächst
                auf die Klausurbögen. Wenn Sie dann mit
                Ihren Antworten zufrieden sind, übertragen Sie diese auf das
                Antwortblatt. Bitte knittern oder verschmutzen Sie dieses
                Blatt nicht.}
            \item {\Large \sc\bfseries\underline{{No Paper In or Out
                    Rule:}}}
              \EN{Hand in all paper, including optional scrap paper, you received during the
                exam. Not handing in your papers results in no
                grade. All papers are put to your name and hence identifiable!}
              \NL{Al het ontvangen papier (inclusief optioneel kladpapier) kompleet inleveren. Niet
                ingeleverde papieren betekent geen cijfer. Alle
                papieren zijn  op naam gesteld en identificeerbaar!}
              \DE{Bitte alle erhaltenen Papiere, inklusive
                  optionelles Schmierpapier, komplett
                  abgeben. Nicht abgebene Papiere bedeutet keine
                  Note. Alle Papiere
                  ist personalisiert und damit identifizierbar!}
            \end{itemize}%
            
            
          \end{minipage}
          \vspace{3mm}
        }
      }
    }
    %% student identification
    \put(-10,28){
      \fbox{\framebox[\frontpageboxwidth][c]{
          \begin{minipage}{.85\frontpageboxwidth}
            \bf
            \vspace{5mm}
            \NL{Naam student}\DE{Name Student}%
            \EN{Name Student}: {\Large\sf \name~\surname}% \dotfill

            \vspace{5mm}
            \NL{Studentnummer}\DE{Studentnummer}\EN{Student number}:
            {\Large \texttt{\id} }%\dotfill 

            \NL{Klas}\DE{Klasse}\EN{Class}:  \Doelgroep\\
          \end{minipage}
        }}
    }
  \end{picture}
  \clearpage
} % coverpage
% einde voorblad

\newcommand\answerFormHead{
  \setlength\parindent{0pt}
  \begin{picture}(195,63)
    \put(25,85){\begin{minipage}{\linewidth}
        \begin{center}
          {\large\bf\sf \DE{Anwortblatt}\NL{Antwoordblad}\EN{Answer form}}
        \end{center}
      \end{minipage}
    }

    \put(0,40){\begin{minipage}{.55\linewidth}
%      {\textbf{\sf{}Studentnummer}}\\
        \AMCcode{snummer}{7}
      \vspace{.3\baselineskip}
    \end{minipage}}

    \put(79,50){\fbox{    
        \begin{minipage}{.45\linewidth}\sf
          \DE{Nachname und Vorname-,}\NL{Achternaam en
            voornaam}:\surname, \name\\

          Studentnummer:\id \\
          \vspace*{2mm}
        \end{minipage}
    }
  }
  \put(79,20){\begin{minipage}{.45\linewidth}
      \sf\EN{Encode your student number in the boxes on the left. Make
      all boxes on this page that apply completely black with pencil or felt tip.}
    \DE{Tragen Sie ihre studentnummer in den Kästchen auf der
      linke Seite. Machen Sie die Kästchen auf dieser Seite die zutreffen völlig
      Schwarz.}
    \NL{Vul je studentnummer in de hokjes aan de linkerzijde in. Maak
      alle juiste hokjes op deze pagina volledig zwart.}
      
    \end{minipage}
  }
 \put(0,10){\rule{1\linewidth}{1pt}}
  \end{picture}

  \vspace{-12mm}
}

\newcommand\AnswerFormHeadAlt{%
  \begin{picture}(195,0)(0,0)
  \put(0,25){\large\bf Answer sheet \examcode{} \NL{datum}\DE{Datum}\EN{Date} \examdate}
  \put(0,5){  \namefield{\fbox{    
      \begin{minipage}{.85\linewidth}
        Name: {\Large{\bf \surname{}, \name{}} \hfill{}\texttt{\lang}~student number \textsf{\id}}
        \vspace*{1mm}
      \end{minipage}
    }}}
\end{picture}
}
      
%% AMC tweeking
\AMCformVSpace=1ex
\AMCformHSpace=0em
%\AMCboxDimensions{size=2ex,down=.4ex,rule=.5pt}
\setlength{\columnseprule}{.3pt}


%%% Local Variables: 
%%% mode: latex
%%% TeX-master: t
%%% End: 
